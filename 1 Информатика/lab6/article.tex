\begin{center}
\vspace*{0.1cm}
\huge\textbf{Решения задач}\\
\Large\textbf{М436-М440; Ф448; Ф450-Ф452}
\end{center}

\noindent
\begin{minipage}[t]{0.31\textwidth}
\textbf{M436.}\qquad\textit{Дано 20 чисел $a_1, \\ a_2, \ \dots \ , \ a_{10}, \ b_1, \ b_2, \ \dots \ , b_{10}$. \\Докажите, что множество \\ из 100 чисел (необязательно \\ различных) $a_1 + b_1, \ a_1 + \\ + \ b_2, \ \dots \ , \ a_{10} + b_{10}$ можно \\ разбить на 10 подмножеств, \\ по 10 чисел в каждом, так,  \\чтобы сумма чисел в каждом \\ подмножестве была одной и \\ той же.}
\end{minipage}
\hfill
\begin{minipage}[t]{0.67\textwidth}
Запишем наши 100 чисел чисел в квадратную таблицу так, как изображено  на рисунке 1; на пересечении $i$-й строки и $j$-го столбца поставим число $a_i + b_j$. 
Образуем теперь 10 подмножеств так, как показано на рисунке 2 (на рисунке клетки-числа, относящиеся к одному и тому же подмножеству, обозначены одной и той же цифрой).
Легко видеть, что в каждом столбце (и каждой строке) есть представители всех подмножеств, так что индексы $i$ и $j$ чисел $a_i + b_j$, входящих в каждое из подмножеств, принимают все значения от 1 до 10 (ровно по одному разу). Поэтому сумма чисел в каждом из подмножеств одна и та же: $a_1 + a_2 + \dots + a_{10} + b_1 + b_2 + \dots + b_{10}.$
\begin{flushright}
\textit{С. Берколайко}
\end{flushright}
\end{minipage}

\section*{}
\noindent
\begin{minipage}[t]{0.4\textwidth}
\begin{center}
\renewcommand{\arraystretch}{2.6}
\setlength{\tabcolsep}{4.2pt}
\begin{tabular}{c|c|c|c|c|c}
& $b_1$ & $b_2$ & $b_3$ & $\dots$ & $b_{10}$ \\
\hline
$a_1$ & $a_1 + b_1$ & $a_1 + b_2$ & $a_1 + b_3$ & $\dots$ & $a_1 + b_{10}$ \\
\hline
$a_2$ & $a_2 + b_1$ & $a_2 + b_2$ & $a_2 + b_3$ & $\dots$ & $a_2 + b_{10}$ \\
\hline
$\vdots$ & $\vdots$ & $\vdots$ & $\vdots$ & $\vdots \vdots \vdots$ & $\vdots$ \\
\hline
$a_{10}$ & $a_{10} + b_1$ & $a_{10} + b_2$ & $a_{10} + b_3$ & $\dots$ & $a_{10} + b_{10}$ \\
\hline
\end{tabular}
\end{center}
\vspace{0.1cm}
\textbf{Рис. 1.}
\end{minipage}
\hfill
\begin{minipage}[t]{0.45\textwidth}
\begin{center}
\renewcommand{\arraystretch}{1.3}
\setlength{\tabcolsep}{7.2pt}
\begin{tabular}{|c|c|c|c|c|c|c|c|c|c|}
\hline
1 & 2 & 3 & 4 & 5 & 6 & 7 & 8 & 9 & 10 \\ \hline
10 & 1 & 2 & 3 & 4 & 5 & 6 & 7 & 8 & 9 \\ \hline
9 & 10 & 1 & 2 & 3 & 4 & 5 & 6 & 7 & 8 \\ \hline
8 & 9 & 10 & 1 & 2 & 3 & 4 & 5 & 6 & 7 \\ \hline
7 & 8 & 9 & 10 & 1 & 2 & 3 & 4 & 5 & 6 \\ \hline
6 & 7 & 8 & 9 & 10 & 1 & 2 & 3 & 4 & 5 \\ \hline
5 & 6 & 7 & 8 & 9 & 10 & 1 & 2 & 3 & 4 \\ \hline
4 & 5 & 6 & 7 & 8 & 9 & 10 & 1 & 2 & 3 \\ \hline
3 & 4 & 5 & 6 & 7 & 8 & 9 & 10 & 1 & 2 \\ \hline
2 & 3 & 4 & 5 & 6 & 7 & 8 & 9 & 10 & 1 \\ \hline
\end{tabular}
\end{center}
\vspace{0.04cm}
\textbf{Рис. 2.}
\end{minipage}

\noindent
\begin{minipage}[t]{0.31\textwidth}
\textbf{М437.}\quad\textit{Докажите, что \\ нечетное число, являющееся \\ произведением n различных \\ простых множителей, можно \\ представить в виде разности \\ квадратов двух натруальных \\ чисел ровно $2^{n-1}$ различными \\ способами.}
\end{minipage}
\hfill
\begin{minipage}[t]{0.67\textwidth}
Представлению нечетного числа a в виде разности двух квадратов $a = x^2 - y^2$ 
соответствует его разложение в произведение двух множителей $a = (x - y)(x + y)$.
Это соответствие взаимно однозначно: по каждому разложению a = rq (где r<q) из системы уравнений $x - y = r$,
$x + y = q$ однозначно определяются $x = (r + q)/2$ и $y = (q - r)/2$ (поскольку a нечетно, 
оба множителя r и q тоже нечетны). Выясним, сколькими способами можно разложить число 
$a = p_1p_2 \dots p_n$ , где $p_1, p_2, \dots, p_n$ - различные простые множители, в произведение двух натуральных 
чисел: $a = rq$. Из n множителей $p_1, \dots, p_n$ можно $2^n$ способами выбрать некоторое (в частности, пустое) подмножество - произведение этих множителей даст r, а произведение остальных - q (пустое подмножество соответствует единице). Таким образом, всех представителей $a = rq$ существует $2^n$, а таких, в которых $r < q$, - вдвое меньше: $2^{n-1}$.
\begin{flushright}
\textit{О. Гончарик, С. Сергей}
\end{flushright}
\end{minipage}

\vspace{0.1cm}
\noindent
\begin{minipage}[t]{0.31\textwidth}
\textbf{М438.}\quad\textit{В данный сегмент \\ вписысываются всевозможные \\ пары касающихся окружностей. \\ Для каждой пары окружностей \\ через точку каасания про-}
\end{minipage}
\hfill
\begin{minipage}[t]{0.67\textwidth}
\makebox[0pt][l]{\raisebox{5ex}{$\blacklozenge$}}\\[-3ex] 
Докажем, что все эти прямые проходят через точку M - середину дуги сегмента, дополняющего данный сегмент до круга. Обозначим границу этого круга через $\gamma$ (рис 3).

\qquadЧерез K обозначим точку пересечения диаметра MN окружности $\gamma$ с хордой AB данного сегмента. Пусть $\gamma_1$ и
\end{minipage}
\fancyfoot[L]{30}

\newpage
\thispagestyle{empty}
\vspace{0.1cm}
\noindent
\begin{minipage}[t]{0.31\textwidth}
\textit{где $a_1, a_2, \dots, a_n$ - действи- \\ тельные, $k_1, k_2, \dots, k_n$ - на- \\ туральные числа) имеет не \\ более n положиетльных кор- \\ ней. }

\qquad\textit{в) Докажите, что \\ уравнение \\ $ax^k(x+1)^p+bx^l(x+1)^q+$}

\hspace{2cm}
\textit{
    $+cx^m(x+1)^r=1$ \\
    (где a, b, c - действительные, \\
    k, l, m, p, q, r - натуральные \\
    числа) имеет не более 14 \\
    положительных корней.
}
\vspace{0.2cm}

\includegraphics[width=0.8\linewidth]{graph.png}

\vspace{0.1cm}
\textbf{Рис. 5.}
\end{minipage}
\hfill
\begin{minipage}[t]{0.71\textwidth}
части уравнения $(l')$ на $(-\overline{a}_nx^kn^-1)$, получим уравнение $b_1x^{k_1-k_n} + b_2x^{k_2-k_n} + \dots + b_{n-1}x^{k_{n-1}-k_n} = 1$\hspace{10.63cm} $(2)$

\vspace{0.2cm}
(имеющее более $n-1$ положительных корней). Продиффернцировав обе части уравнения $(2)$, получим уравнение $\overline{b}_1x^{k_1-k_n-1} + {\overline{b}_2x^{k_2-k_n-1} + \dots}$ 

\hspace{8.45cm}
$+\overline{b}_{n-1}x^{k_{n-1}-k_n-1} = 0$,
\hspace{1.2cm} $(2')$

\vspace{0.2cm}
имеющее более $n-2$ положительных корней. Поделив обе части $(2')$ на $(-\overline{b}_{n-1} \\ x^{k_{n-1}-k_n-1}$), получим уравнение $c_1x^{k_1-k_{n-1}} + c_2x^{k_2-k_{n-1}} + \dots + c_{n-2}x^{k_{n-2}-k_{n-1}} = 0$, $(3)$ имеющее более $n-2$ положительных корней.

\hspace{1cm}
Проделав указанные действия $n-1$ раз, мы придем к уравнению

\begin{center}
$\alpha x^m = 1, (m=k_1-k_2)$ ,
\end{center}
которое, в силу сделанного предположения относительно уравнения $(1)$, должно иметь более одного положительного корня. Но это невозможно; значит, исходное уравнение $(1)$ не может иметь более n положительных корней. Утверждение задачи б) доказано.

\hspace{1cm}
Перейдем к задаче в). Нам понадобится следующий факт.

Пусть $P_m(x)$ - многочлен от x степени m. Тогда производная выражения $x^k(x+1)^p P_m(x)$ имеет вид $x^{k-1}(x+1)^{p-1} P_{m+1}(x)$, где $P_{m+1}(x)$ - многочлен от x степени m+1 (k и p - любые действительные числа). \\
Действительно,
\vspace{0.1cm}

$\begin{aligned}
    (x^k(x+1)^p P_m(x))' &= kx^{k-1}(x+1)^pP_m(x) + \\
    &\hspace{1cm} + px^k(x+1)^{p-1}P_m(x) + x^k(x+1)^pP'_m(x) = \\
    &\hspace{1cm} = x^{k-1}(x+1)^{p-1} \big( k(x+1)P_m(x) + pxP_m(x) + \\
    &\hspace{2.6cm} + x(x+1)P'_m(x) \big) = x^{k-1}(x+1)^{p-1}P_{m+1}(x)
\end{aligned}$
\end{minipage}
